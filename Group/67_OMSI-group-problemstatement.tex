\documentclass[10pt,a4paper,onecolumn,draftclsnofoot]{IEEEtran}
\usepackage[utf8]{inputenc}
\usepackage{amsmath}
\usepackage{amsfonts}
\usepackage{amssymb}
\usepackage[margin=0.75in, letterpaper]{geometry}
\author{Travis Gray, Thu Hoang, Tasnia Kabir, Sergio Ortega-Rojas\\Senior Capstone Design, CS461 F18}
\title{Problem Statement}
\date{}
\begin{document}
\maketitle
\begin{abstract}
Our team is working alongside and under the direction of our client, Rebecca Reilly, a technology educator at the Oregon Museum of Science and Industry (OMSI). OMSI has need for a new curricula for teenagers when they attend classes or camps about technology. Our task is to develop this new curriculum using the background knowledge that we have gained from our studies and experiences. Our first steps towards solving this problem include doing some preliminary research about the existing programs that OMSI has set in place, the types of topics and projects that students find interesting and want to work with, and how we can use all of the information that we have to construct a new and exciting curriculum that OMSI can use for weeks, months and hopefully years to come. We provide an in-depth overview of our project and the set of deliverables to our client.
\end{abstract}
\newpage
\section*{Problem}
As of right now, our client only uses one curriculum with all of her groups, but that is something that she wants to change. The problem that she hopes for us to solve is the actual ideation and creation of another curriculum that she could ideally starting using and teaching as early as the summer or fall of 2019. Our deliverables would be either some sort of lesson plan, set of activities and/or a ‘sandbox’ to be presented to the students who attend classes at OMSI. Because of these deliverables, we as a team will have to choose an idea for a potential project and then work backwards to break it down and create it into separate lesson plans for beginners to learn from. Since our client personally has less experience with and knowledge in the technology and software field, she hopes that our strengths and fundamental understanding of computer science could help create a new and exciting project (or set of projects) that would both intrigue and educate her students. However, she has been an educator for many years and therefore will be more than able to help us with the actual structuring of content and creation of lesson plans. The current projects that these students have been working on have been more hardware based, using things like Arduinos, with some introductory coding. For example, this past year the teens worked on creating electronic textiles, so they ended up making things such as blankets or jackets that would light up when they detect a change in temperature or light. For our proposed curriculum, we are hoping to take it in a different direction, possibly more software based so that the kids get more into the actual programming, problem solving, logic, and design thinking.

\section*{Solution/Tools and Methods}

\subsection*{Solution}
Our proposed solution is to create a curricula that is more focused on software, since we are computer science students rather than electrical engineering students. Creating a more inclusive learning environment for the students to ensure that it is not geared to one gender over another. While also giving the students an actual problem they that have to solve and figure out. Finally, creating a curriculum that is fun and that the kids are able to learn from and remember.

An example of one of our proposed solution is creating a talking creature would be able to take some input from the user and then return pre-loaded replies that are written by the teenagers as well. The teenagers would have creativity in creating the look of their creatures as well as the input and replies. There would be some instruction to teach the teenagers how code in multiple different languages and learning how different languages can be combined to create a finished product. 

The proposed plan would walk the teenagers through a modified software development cycle in which they first design their creature, then implement their creature, and lastly test their design to make sure that it is working in the way it is meant to. By doing so, it would show a real life application of how software is created in industry while they are learning how to code.

The software needed with this is a simple IDE. The project would be saved onto the laptop, allowing the teenager (or pair of teenagers depending on how it would be structured) to be assigned a computer and work on the same computer until the end of the class. The languages taught would be HTML, CSS, and JavaScript. These languages are relatively easy for a beginner to learn these languages and allows the teenagers to learn how a simplistic web application can be brought to life.

With the main concept of the software development cycle in real life through the steps of designing, implementing, and testing of their creature. Along with the knowledge of three different languages and how they are able to communicate together.
\subsection*{Method}
In the introduction of the course, the teenagers will design what their creature will look like and say. The teenagers will draw the creature on a piece of paper. The design will include how the creature will look like from size to the color of the creature through the use of colored pencils. design would stay in the classroom until the end of the camp so they won't lose it for implementing their creatures later. The creature would  After they finish designing their creatures, they would then design what they would like their creatures to say/react to when given an input by a user. This would be written on the same paper that the creature is drawn on so it will be all in one place for implementation.

After the teenagers are done with the designing aspect of the project, there would then be a quick lesson on the first language HTML. They would learn how to create a HTML webpage on the IDE and then implement said HTML by starting out with a page with the name of their creature or whatever they would like. They would have time to explore and play around with the web page they just created. 

After they finish learning about HTML, there would then be a lesson on CSS and how to use CSS in tandem of HTML. Here, they will create their creature. This might take a little bit of time and if they thing that their creature won't be able to work, or if it is too difficult to do, then they must redesign their original creature on a different paper.

The last thing to learn would be JavaScript. This would be to implement the last part of the creature in where it will "talk" by printing out a speech bubble when there is input from a user. This will most likely be the most labor intensive part of the project since it must be written and able to work together with HTML and CSS.

The teenagers will then write test cases that showcases that their creatures will work like they say it will. In order to test, they will do manual testing with the test cases they write down. When it's done, they are complete with their project. After the camp is done, the creatures will be saved and uploaded to a part of the OMSI server where the teenagers will be able to access their creatures whenever they want to.

\section*{Performance Metrics}
Since our project is more open-ended, our performance metrics are not as objective as some of the other technical projects might be. In this case, I think that our most valuable performance metric is the feedback that we receive not only from our client, but also from the students that actually participate in the course. This feedback includes how our client and students feel about the course, which parts they would keep, which parts they would change, and any additional insight that they are able to offer us. Based on our meeting with our client, we were not given any other explicit requirements other than creating a curriculum that the students would enjoy and learn from that works with our given constraints. As of right now it feels as though the project is much more open ended and we have the freedom to take it in whichever direction we choose. This is something that we will continue to inquire about and track over the course of the year through weekly meetings with our group, our TA, and our client.

\section*{Conclusion}
With student feedback and proper collaboration, we hope to bring a curriculum to the Oregon Museum of Science and Industry that will inspire teens to understand the technology of tomorrow.

\end{document}