\documentclass{article}
\usepackage[utf8]{inputenc}

\title{CS 461: Problem Statement}
\author{Tasnia Kabir}
\date{October 2018}

\usepackage{natbib}
\usepackage{graphicx}

\begin{document}

\maketitle

\section{Abstract}
For our senior capstone project, my team and I are working alongside and under the direction of our client, Rebecca Reilly, a science and now technology educator at the Oregon Museum of Science and Industry (OMSI). Currently, Rebecca inherited and now runs a program at OMSI that gives children, typically ranging from the ages 13-16, a hands-on opportunity to work with and learn about some cool technology, hardware, and software. During their time at OMSI, these teens come together to learn various aspects of technology and create something that they can walk away with and show off all of their hard work. At this time, Rebecca teaches three types of courses: one summer course that goes on for ten weeks where the kids meet for 2 hours a week, another summer course that only goes on for one week (5 days) where the kids meet for 7 hours a day, and one more course that takes place during the school year with homeschooled students. She is now looking for an additional curriculum to teach all of these groups of students, and that is where we come in. Our task is to develop this new curriculum using the background knowledge that we have gained from our studies and experiences. Our first steps towards solving this problem include doing some preliminary research about the existing programs that OMSI has set in place, the types of topics and projects that students find interesting and want to work with, and how we can use all of the information that we have to construct a new and exciting curriculum that OMSI can use for weeks, months and hopefully years to come.

\section{Problem}
As of right now, our client only uses one curriculum with all of her groups, but that is something that she wants to change. The problem that she hopes for us to solve is the actual ideation and creation of another curriculum that she could ideally starting using and teaching as early as the summer or fall of 2019. Our deliverables would be either some sort of lesson plan, set of activities and/or a ‘sandbox’ to be presented to the students who attend classes at OMSI. Because of these deliverables, we as a team will have to choose an idea for a potential project and then work backwards to break it down and create it into separate lesson plans for beginners to learn from. Since our client personally has less experience with and knowledge in the technology and software field, she hopes that our strengths and fundamental understanding of computer science could help create a new and exciting project (or set of projects) that would both intrigue and educate her students. However, she has been an educator for many years and therefore will be more than able to help us with the actual structuring of content and creation of lesson plans. The current projects that these students have been working on have been more hardware based, using things like Arduinos, with some introductory coding. For example, this past year the teens worked on creating electronic textiles, so they ended up making things such as blankets or jackets that would light up when they detect a change in temperature or light. For our proposed curriculum, we are hoping to take it in a different direction, possibly more software based so that the kids get more into the actual programming, problem solving, logic, and design thinking.

\section{Proposed Solution}
Our solution to our client’s problem is to come up with a completely new curriculum that can be implemented as soon as possible. As we attempt to come up with this new curriculum, some of our goals include, but are not limited to, the following: making the project more software-based, since we are computer science students rather than electrical and computer engineering students; emphasizing the importance of the designing and brainstorming portion of the process; creating a more inclusive learning environment for the students to ensure that it is not gendered; giving the students an actual problem they that have to solve and figure out; and lastly, creating a curriculum that is fun and that the kids are able to learn from and remember. Some of the tools and resources that we will have include the following: our client, Rebecca, and her experience with teaching and structuring curriculum; twelve 11” laptops (Dell Latitude 3180, purchased this year); some random electronics such as LED lights, breadboards, resistors and such; and lastly, a budget of \$300 to cover any additional supplies or software that we might need for the course. A couple of other constraints that we have include the actual time that we are able to spend with the students, the fact that we cannot give any homework or group work outside of class time, and the assumption that students have no prior knowledge with coding. The summer program allocated about 25-30 hours for programming, whereas the school year program has about 20 hours. In general, the summer program is much more relaxed and laid back because it has the summer camp feel, while the school year program is structured much more like a class in school.

\section{Performance Metrics}
Since our project is more open-ended, our performance metrics are not as strict or hard as some of the other, more technical projects’ might be. In this case, I think that our most valuable performance metric is the feedback that we receive not only from our client, but also from the students that actually partake in the course. This feedback includes how our client and students feel about the course, which parts they would keep, which parts they would change, and any additional insight that they are able to offer us. Based on our meeting with our client, we were not given any other hard requirements other than simply creating a curriculum that the students would enjoy and learn from that works within all of our given constraints. As of right now it feels as though the project is much more open ended and we have the freedom to take it in whichever direction we choose, while Rebecca would help us pivot along the way. This is something that we will continue to inquire about and track over the course of the year through weekly meetings with our group, our TA, and our client.


\end{document}
