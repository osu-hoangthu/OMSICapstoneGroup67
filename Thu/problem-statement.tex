\documentclass[10pt, letterpaper]{article}
\usepackage[utf8]{inputenc}

\title{OMSI Teen Curricula Problem Statement}
\author{Thu Hoang}
\date{\today}

\begin{document}
\maketitle
\begin{abstract}
    This article states a problem that OMSI has which is the need for a new curricula for teenagers when they attend classes or camps about technology. Currently, there is one program that is centered around technology, but OMSI would like another program to add to their classes and camps. A proposed solution is a project to create a talking creature through the use of HTML, CSS, and JavaScript on a web page. After completing the project, a teenager should be able to apply their learning to create their own web page and use the problem solving skills they learned in their own life.
    
\end{abstract}
\newpage
\section{Problem}
    OMSI currently has camps in which teenagers from age 13-15 learn and create technology with their own hands. Currently, there is a need for new curricula to add into their current rotation to be used by Fall 2019. The future curricula would be a hands-off teaching approach where teenagers learn how to code and build with their own hands. The current curricula involves the teenagers to "solve a problem", leaving the problem open ended and a way to have teenagers to find problems in the current world and to find ways to solve them. What my group and I are tasked with is to create a curricula that teenagers would be be able to hold their interest for 20-35 hours while learning how to code.
    
\section{Solution/Tools and Methods}
\subsection{Solution}
    The solution is to create a curricula that would involve instant gratification to the teenagers involved in the camps. The curricula would be simplistic enough for an instructor with little to no knowledge of the technology but difficult enough for teenagers to think critically and apply knowledge they not only learn in the classroom, but from their outside resources such as the internet if need be. The curricula would be mostly coding with the use of a sandbox application to create a talking creature. The talking creature would be able to take some input from the user and then return pre-loaded replies that are written by the teenagers as well.
    
    The teenagers would have creativity in creating the look of their creatures as well as the input and replies. There would be some instruction to teach the teenagers how code in multiple different languages and learning how different languages can be combined to create one finished product. In the end, their project can be uploaded onto the OMSI server so it can be accessed by anybody. There would be nothing that the student can bring back, but they are able to view their project and be able to show what they've done to a wider audience that could also be showcased by the museum.
    
    The proposed plan would walk the teenagers through a modified software development cycle in which they first design their creature, then implement their creature, and lastly test their design to make sure that it is working in the way it is meant to. By doing so, it would show a real life application of how software is created in industry while they are learning how to code.
    
    The software needed with this is a simple IDE. In the implementation, there is a soft of web page implementation that is done, but the use of a web server is not necessarily needed in order to implement the lesson plan. The project must be saved onto the laptop, meaning that the teenager (or pair of teenagers depending on how it would be structured) would be assigned a computer and will work on said computer to the end of the camp. The languages taught would be HTML, CSS, and JavaScript. These languages are relatively easy for a beginner to learn these languages and would allow the teenagers to learn how a simplistic web page is made and how languages can be put together in real life.
    
    The major take away from this camp is how software is developed in real life through the steps of designing, implementing, and testing their creature. Another take away from this camp would be the knowledge of three different languages and how languages are able to communicate together.
    
\subsection{Method}
    In the introduction of the course, the teenagers will design what their creature will look like and say. The teenagers will draw the creature on a piece of paper. The design will include how the creature will look like from size to the color of the creature through the use of colored pencils. design would stay in the classroom until the end of the camp so they won't lose it for implementing their creatures later. The creature would  After they finish designing their creatures, they would then design what they would like their creatures to say/react to when given an input by a user. This would be written on the same paper that the creature is drawn on so it will be all in one place for implementation.
    
    After the teenagers are done with the designing aspect of the project, there would then be a quick lesson on the first language HTML. They would learn how to create a HTML webpage on the IDE and then implement said HTML by starting out with a page with the name of their creature or whatever they would like. They would have time to explore and play around with the web page they just created. 
    
    After they finish learning about HTML, there would then be a lesson on CSS and how to use CSS in tandem of HTML. Here, they will create their creature. This might take a little bit of time and if they thing that their creature won't be able to work, or if it is too difficult to do, then they must redesign their original creature on a different paper.
    
    The last thing to learn would be JavaScript. This would be to implement the last part of the creature in where it will "talk" by printing out a speech bubble when there is input from a user. This will most likely be the most labor intensive part of the project since it must be written and able to work together with HTML and CSS.
    
    The teenagers will then write test cases that showcases that their creatures will work like they say it will. In order to test, they will do manual testing with the test cases they write down. When it's done, they are complete with their project. After the camp is done, the creatures will be saved and uploaded to a part of the OMSI server where the teenagers will be able to access their creatures whenever they want to.

\newpage
\section{Performance Metrics/Criteria}
    The criteria that we will use as a measurement would be the level of learning that our teenage participants would have. What this would mean is that the teenagers that participate in this lesson plan would be able to use what they learned and apply it to create their own web pages in the future if they choose to do so. In the end of the curricula, the teenagers should have created a memorable creature/web page and would be able to use the information that they learned from the program in other aspects of their life. For example, they would have the tools to create a web page if they would want to, they could also use the problem solving skills that they learned and apply it to other aspects of their life.

\end{document}
