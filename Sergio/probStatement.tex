\documentclass[10pt,a4paper,onecolumn,draftclsnofoot]{IEEEtran}
\usepackage[utf8]{inputenc}
\usepackage{amsmath}
\usepackage{amsfonts}
\usepackage{amssymb}
\usepackage[margin=0.75in, letterpaper]{geometry}
\author{Sergio Ortega-Rojas\\Senior Capstone Design, CS461 F18}
\title{Problem Statement}
\date{}
\begin{document}
\maketitle
\begin{abstract}
This document provides details about project 67 for CS461, a project involved with creating a set of lessons involving a high interest topic in Computer Science. The project is quite open ended and leaves a suitable amount of interpretation to the group to provide that lesson plan. As such, a more in-depth definition/description of the project is also available to provide a clear and definite picture of the problem. In the document I will also talk about two different approaches that will be considered as potential solutions along with our expectations that we will measure to ensure a successful approach to the problem.
\end{abstract}
\newpage
\section*{Problem Description}
Project 67 has project title name, as such in this document it will be referred to as OMSI Curriculum. OMSI, Oregon Museum of Science and Industry, provides a variety of camps and classes throughout the year to range of different age groups. Our team is tasked with targeting the high school demographic, with an introduction into one of the high interest topics in today's Computer Science world. We are to research what is currently trending in today's youth to bridge the gap between concept and application. After a meeting with the client, we have been given a set of constraints as well. The client has stated the specifics: 12 Laptops are available, internet connection is spotty, Instructor has a limited amount of knowledge with Computer Programming languages, and a limited amount of software available. There will also be more in the future, however, these are the ones that came up during the meeting with the client.

Currently OMSI provides a curriculum for students that involves textiles and sensors. In it the students are given instructions on how to sew, and then are also introduced to some coding knowledge during their initial 20 hours. Afterwards they are given hands off time to attempt to solve a problem using their new found skills. Students have made a variety of different items, anything from a comfort buddy that lights up to a blanket that lights up when a certain temperature is reached. The students are also given the opportunity to take their projects home with them. This curriculum is made to be used in either a 10 week format or a five day crash course during summer. This gives a picture of what we are trying to capture with the project.

As a group we are attempting to create a flexible curriculum without a large amount of technical overhead, to prevent the instructor from being unable to fully support the students. We must also take into account the lack of network availability during summer lessons, as the class is moved to a different location in OMSI. Usually lessons are done in Summer and Fall which means limited opportunity to perform work outside due to the variable weather. Since the lessons involve teens from 12-15, while direct instruction is a simple approach it is not always effective, as noted by our client, teens want their own opportunity to do their own thing.

\section*{Proposed Solution}
For this project we need a solution that works with the given constraints while also being in similar line with the parallel curriculum. If we do propose a high level solution, then we must look to train the instructor or provide a detailed enough teaching plan that covers any bugs that may be encountered. This means that most of the skills that are involved with this curriculum are quite low level. The instructor doesn't have a background in Computer Science, so when we look to implement a strictly code based activity it may be detrimental. Our solution must be able to be inherited by the next in line, regardless of their background.

As such, I propose to create a curriculum that involves the students creating their own raspberry pi computers with a custom User Interface created by us that also gives them a certain amount of flexibility. Then we can have them use their creation to play games with a central computer, we provide the framework that allows them to modularize their computers. That way every teen's computer is unique to themselves, and provides an entry to be able to further improve their computer on their own time. The cost of entry is also cheap as each individual unit costs only thirty five dollars for the latest board, and ten dollars for the most basic wireless model. By providing a variety of different connections and our own creations to give them an idea on how far they can be used. 

Since the pis can be run isolated and don't require a ton of power the direction that can be taken with them is limitless. Also if we provide the Operating System that is to be loaded on the pis then we can provide high level tools for them to use. It also addresses the concerns that we would have with the instructors, as everything would be detailed and could be supported with future patches. Since Linux is an open source OS, we could branch off from one of the kernels and release it for other institutions to use. This solution keeps the spirit of the other curriculum, giving the kids an opportunity to explore Computer Science by empowering them with the skills to create.

\section*{Performance Metrics}
The metrics by which we will measure our success, is the feedback that we receive in our initial focus group that we will test our first run of the curriculum. Should the initial group of teens enjoy the program, then we can focus on the aspects that they enjoyed along with addressing parts that weren't quite as enjoyable as we anticipated. A successful project will involve a full 5 day 35 hour curriculum that provides the teens the opportunity to explore Computer Science and potentially take their projects home. It must also ensure that the instructor is able to teach to the teens without input from our group, a true sandbox experience that can be taken anywhere. Along with a full set of lecture notes that can be taught by anyone, regardless of their backgrounds. We will have a few opportunities in February to test our project and make sure that we are in line with expectations. Our client is expecting it to be completed by May, and become a part of their rotation. 

\end{document}